\documentclass[11pt]{article}

\usepackage[left=2.5cm,top=2.5cm,right=2.5cm,bottom=2.5cm,letterpaper,twoside]{geometry}

\usepackage{amsmath}

\setlength{\parindent}{0pt}

\begin{document}

I is given by:
\begin{equation*}
  I = P_1 \cdot I + P_2 \cdot (\cos{2i_1} \cdot Q - \sin{2i_1} \cdot U)
\end{equation*}
since $I = 1$, this means
\begin{equation*}
  I = P_1 + P_2 \cdot (\cos{2i_1} \cdot Q - \sin{2i_1} \cdot U) \equiv P_1 + P_2  \cdot \gamma
\end{equation*}
where
\begin{equation*}
\gamma = \cos{2i_1} \cdot Q - \sin{2i_1} \cdot U
\end{equation*}

$P_1$ and $P_2$ are functions of $(\mu,\nu)$ where $\mu=\cos{\theta}$. So for a given $(\mu,\nu)$, $P_1$ and $P_2$ are constants. In this case, I is maximized when $P_2\cdot\gamma$ is maximized. If $P_2 < 0$, this occurs when $\gamma$ is minimized, and if $P_2 > 0$, this occurs when $\gamma$ is maximized. So the question is, what is the range of values that $\gamma$ can take?\\

The variables inside $\gamma$ are correlated: $\sin{2i_1} = \sqrt{1.-\cos^2{2i_1}}$ and $|U| < \sqrt{1. - Q^2}$.\\

I wrote a small program that chooses a random $\cos{2i_1}$ between -1 and 1, and calculates the corresponding $\sin{2i_1}$. Then a random Q is sampled between -1 and 1, and U is randomly sampled between $-\sqrt{1. - Q^2}$ and $\sqrt{1. - Q^2}$:
\begin{verbatim}
program gamma_range
  
  implicit none
  integer,parameter :: p = 8
  real(p) :: cosx,sinx,q,u
  real(p) :: gamma,gamma_min=0._p,gamma_max=0._p
  integer :: iter
  
  do iter=1,100000000
    call random_number(cosx) ; cosx = cosx * 2._p - 1._p
    sinx = sqrt(1._p-cosx*cosx)
    call random_number(q) ; q =  q*2._p-1._p
    call random_number(u) ; u = (u*2._p-1._p) * sqrt(1._p-q*q)
    gamma = cosx*q-sinx*u
    if(gamma > gamma_max) gamma_max = gamma
    if(gamma < gamma_min) gamma_min = gamma
  end do
  
  print *,'Min = ',gamma_min
  print *,'Max = ',gamma_max

end program gamma_range
\end{verbatim}

Which produces the following results
\begin{verbatim}
$ ./a.out
Min =  -0.999999580188680     
Max =   0.999992504333863   
\end{verbatim}

Therefore it seems that $\gamma\in[-1:1]$. If $P_2 < 0$, the maximum value of $P_2\cdot\gamma$ is $-P_2$, and if $P_2 > 0$, the maximum value of $P_2\cdot\gamma$ is $P_2$. So the maximum value of $P_2\cdot\gamma$ is $|P_2|$. This means that the maximum value of I is
\begin{equation*}
I = P_1 + |P_2|
\end{equation*}

Since $P_1 + |P_2|$ is still a function of $\mu$, this means that at a given frequency, the `ceiling' value used in the rejection criterion is the maximum value of $P_1 + |P_2|$ at that frequency. It should therefore be possible to pre-compute the maximum $P_1 + |P_2|$ at each frequency. For the scattering of a photon of a frequency $\nu$, the maximum I for the rejection criterion could then be found from simple interpolation.

\end{document}


