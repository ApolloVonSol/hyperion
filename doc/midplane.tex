\documentclass[11pt]{article}
\begin{document}

\title{}
\author{}
\date{}
\maketitle

\section{TSC envelope}

To find the midplane density of the TSC envelope model, we have to estimate 
The TSC density structure is given by

\begin{equation}
\rho(r, \theta) = \rho_0\left(\frac{r}{R_c}\right)^{-3/2}\,\left(1+\frac{\mu}{\mu_0}\right)^{-1/2} \, \left(\frac{\mu}{\mu_0} + \frac{2\,\mu_0^2\,R_c}{r}\right)^{-1}
\end{equation}

Writing $\gamma\equiv r/R_c$, this is

\begin{equation}
\rho(r) = \rho_0\,\gamma^{-3/2}\,\left(1+\frac{\mu}{\mu_0}\right)^{-1/2} \, \left(\frac{\mu}{\mu_0} + \frac{2\,\mu_0^2}{\gamma}\right)^{-1}
\end{equation}

We now have to find $\mu_0$ for $\mu=0$. The streamline equation is:

\begin{equation}
\mu_0^3 + \mu_0\,(\gamma - 1) - \mu\,\gamma = 0
\end{equation}

If $\mu=0$, this gives

\begin{equation}
\mu_0\,[\mu_0^2 + (\gamma - 1)]= 0
\end{equation}

This has solution $\mu_0 = 0$ and $\mu_0^2 = (1-\gamma)$.

\subsection{Outside $R_c$}

For $\gamma > 1$, neither of these are satisfactory (since $\mu/\mu_0$ will be undefined for the former, and $\mu_0$ is imaginary for the latter. The solution \textit{is} 0, but we need to compute $\mu/\mu_0$ by some other means. We can multiply the streamline equation by $\mu^2/\mu_0^3$, giving

\begin{equation}
\mu^2 + \left(\mu/\mu_0\right)^2\,(\gamma - 1) - \left(\mu/\mu_0\right)^3\,\gamma = 0
\end{equation}

For $\mu=0$, this is 

\begin{equation}
\left(\mu/\mu_0\right)^2\,(\gamma - 1) - \left(\mu/\mu_0\right)^3\,\gamma = 0
\end{equation}

which can be factorized as

\begin{equation}
\left(\mu/\mu_0\right)^2\,\left[(\gamma - 1) - \left(\mu/\mu_0\right)\,\gamma\right] = 0
\end{equation}

This has two unique solutions, $\mu/\mu_0=0$ and $\mu/\mu_0 = (\gamma - 1)/\gamma$. The former is not acceptable, because combined with $\mu_0$ = 0, this gives an undefined density. Therefore, we have:

\begin{equation}
\left\{\begin{array}{l}\mu_0 = 0 \\\displaystyle\mu/\mu_0 = \frac{\gamma - 1}{\gamma}\end{array}\right.\end{equation}

Which gives

\begin{equation}
\rho(r) = \rho_0\,\gamma^{-3/2}\,\left(\frac{2\gamma-1}{\gamma}\right)^{-1/2} \, \left(\frac{\gamma-1}{\gamma}\right)^{-1}
\end{equation}

This can be simplified:

\begin{equation}
\rho(r) = \rho_0\,(2\gamma-1)^{-1/2} \,(\gamma-1)^{-1}
\end{equation}

In cumulative terms, this is

\begin{equation}
\Sigma(r) = \rho_0 \times \left\{ \ln{\left[\frac{\sqrt{2\gamma_1 - 1} - 1}{\sqrt{2\gamma_1 - 1} + 1}\right]} - \ln{\left[\frac{\sqrt{2\gamma_0 - 1} - 1}{\sqrt{2\gamma_0 - 1} + 1}\right]} \right\}
\end{equation}



\subsection{Inside $R_c$}

For $\gamma < 1$, The ratio equation only has one real positive solution, which is 0. However, both solutions for $\mu$, i.e. 0 and $1-\gamma$ are valid. Since the ratio is 0, this means the only solution which does not give an undefined density is $1-\gamma$. Therefore, 

\begin{equation}
\left\{\begin{array}{l}\mu_0^2 = 1-\gamma \\\displaystyle\mu/\mu_0 = 0\end{array}\right.\end{equation}


Which gives

\begin{equation}
\rho(r) = \rho_0\,\gamma^{-3/2}\,\left[\frac{2(1-\gamma)}{\gamma}\right]^{-1}
\end{equation}

i.e.

\begin{equation}
\rho(r) = \frac{\rho_0}{2}\,\frac{\gamma^{-1/2}}{1-\gamma}
\end{equation}

In cumulative terms, this is

\begin{equation}
\Sigma(r) = \frac{\rho_0}{2} \times \left\{ \ln{\left[\frac{\sqrt{\gamma_1}+1}{1-\sqrt{\gamma_1}}\right]} - \ln{\left[\frac{\sqrt{\gamma_0}+1}{1-\sqrt{\gamma_0}}\right]} \right\}
\end{equation}


\end{document}
