\documentclass[11pt]{article}
\begin{document}

\title{Dust sublimation radius}
\author{Thomas Robitaille}
\date{\today}
\maketitle

The factor $W(r)$ describes the probability that a photon, emitted at $r$ in a random direction, is intercepted by the star, which has radius $r_\star$ at the wavelength considered. The factor 1-W(r) is the probability that a photon emitted at $r$ escapes. $W(r)$ is the fraction of the solid angle that is covered by the star, so

$$
W(r) = \frac{1}{2}\left(1-\sqrt{1-(r_{\star}/r)^2}\right)
$$

This factor is called the \textit{geometrical dilution factor}. Photons emitted just above $r_\star$ have a $50$\% chance of being re-absorbed by the star.

...

The radiative equilibrium condition for determining the temperature $T_d$ of a grain is

$$
\int_0^\infty\,\kappa_\nu\,B_\nu(T_d)\,d\nu = \int_0^\infty\,\kappa_\nu\,J_\nu\,d\nu
$$

The left hand side of this equation is the radiative cooling of a grain because the product $\kappa_\nu\,B_\nu(T_d)$ is the thermal emissivity at $\nu$ of a grain at temperature $T_d$. The right hand side is the radiative heating owing to the opacity $\kappa_\nu$ and the ambient radiation field. The quantity $J_\nu$ is the monochromatic mean intensity of the radiation field incident on the grain, and it is the angle average mean of the intensity $i_\nu$. For the optically thin case in which the intensity incident on the grain is direct star light with a uniform intensity $I_\nu=B_\nu(T_\star)$ the mean intensity is given by

$$
J_\nu = W(r) B_\nu(T_\star)
$$

The result is that

$$
T_d^4\,\kappa_{\rm plank}(T_d) = T_\star^4\,W(r)\,\kappa_{\rm plank}(T_\star)
$$

Re-arranging for $W(r)$ gives

$$
W(r) = \frac{T_d^4}{T_\star^4}\frac{\kappa_{\rm plank}(T_d)}{\kappa_{\rm plank}(T_\star)}
$$

Expanding gives

$$
 \frac{1}{2}\left(1-\sqrt{1-(r_{\star}/r)^2}\right) = \frac{T_d^4}{T_\star^4}\frac{\kappa_{\rm plank}(T_d)}{\kappa_{\rm plank}(T_\star)}
$$

$$
1-\sqrt{1-(r_{\star}/r)^2} = 2\,\frac{T_d^4}{T_\star^4}\frac{\kappa_{\rm plank}(T_d)}{\kappa_{\rm plank}(T_\star)}
$$

$$
\sqrt{1-(r_{\star}/r)^2} = 1-2\,\frac{T_d^4}{T_\star^4}\frac{\kappa_{\rm plank}(T_d)}{\kappa_{\rm plank}(T_\star)}
$$

$$
1-(r_{\star}/r)^2 = \left[1-2\,\frac{T_d^4}{T_\star^4}\frac{\kappa_{\rm plank}(T_d)}{\kappa_{\rm plank}(T_\star)}\right]^2
$$

$$
(r_{\star}/r)^2 = 1-\left[1-2\,\frac{T_d^4}{T_\star^4}\frac{\kappa_{\rm plank}(T_d)}{\kappa_{\rm plank}(T_\star)}\right]^2
$$

$$
r_{\star}/r = \left\{1-\left[1-2\,\frac{T_d^4}{T_\star^4}\frac{\kappa_{\rm plank}(T_d)}{\kappa_{\rm plank}(T_\star)}\right]^2\right\} ^ {1/2}
$$

$$
r = r_{\star} \,  \left\{1-\left[1-2\,\frac{T_d^4}{T_\star^4}\frac{\kappa_{\rm plank}(T_d)}{\kappa_{\rm plank}(T_\star)}\right]^2\right\} ^ {-1/2}
$$

\end{document}
